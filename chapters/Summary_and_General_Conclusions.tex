%%%% Time-stamp: <2012-08-20 17:41:39 vk>

%% example text content
%% scrartcl and scrreprt starts with section, subsection, subsubsection, ...
%% scrbook starts with part (optional), chapter, section, ...
\chapter{Summary and General Conclusions}
The term multivariate pattern (MVP) analysis summarizes a range of data
analysis strategies that are highly suitable for studying neural
representations encoded in distributed patterns of brain activity
\citep{haxby_2012}. While there is an ever increasing number of publications
that demonstrate the power of MVP analysis for functional magnetic resonance
imaging (fMRI) data \citep{opdebeeck_2010,freeman_2011,alink_2013,freeman_2013}
with standard resolution data (a voxel size of about \mm{2-3} isotropic), MVP
analysis is especially promising in the context of high-resolution fMRI.
Ongoing technological improvements, such as ultra high-field MRI scanners
(\sevenT\ or higher) have pushed the resolution for fMRI to a level
that is slowly approaching the spatial scale of the columnar organization of
the brain \citep{yacoub_2008,heidemann_2012}. Being able to use fMRI to sample
brain activity patterns at a near-columnar level makes it feasible to employ
MVP analysis with the aim to decode distributed neural representations of an
entire cortical field at a level of detail that is presently only accessible to
invasive recording techniques with limited spatial coverage. However, at this
point, it is unclear which spatial resolution is most suitable for decoding
neural representation from fMRI data with MVP analysis. While higher
resolutions can improve the fidelity of the BOLD signal by, for example,
reducing the partial volume effect \citep{Weibull_2008}, the benefits can be
counteracted by physiological noise (such as inevitable motion) and lower tSNR. 
This raises the question: does the decoding of neural representations continuously improve
with increasing spatial resolution, or is there an optimal resolution for a
given type of representation?

In this study, we aim to address this question for the most frequently employed
MVP analysis technique: a cross-validated classification analysis, where a
classifier is repeatedly trained to distinguish patterns of brain activation
from fMRI data of a set of stimulus conditions, and its prediction accuracy is
evaluated against a left-out data portion \citep{pereira_2009}. Moreover, we
focus on the decoding of the representation of oriented visual gratings in
primary visual cortex. This is likely to be the most extensively studied
paradigm regarding the application of MVP analysis on fMRI data, starting with
the classic studies of  \citet{kamitani_2005} and \citet{haynes_2005}. It was
shown that orientation can be decoded reliably at resolutions ranging from
standard \mm{3} isotropic voxels in the aforementioned studies, to \mm{1}
\citep{swisher_2010}, and that it is possible to directly model orientation
columns in V1 with \sevenT\ fMRI of \mm{0.5$\times$0.5} (in-plane) resolution
\citep{yacoub_2008,ugurbil_2012}. These findings led to a debate on the origin
and the spatial scale of the signals that classifiers can use to learn to
discriminate different orientations
\citep{opdebeeck_2010,swisher_2010,alink_2013,freeman_2013}. To investigate
these questions, the authors typically acquired high-resolution fMRI and
simulated a lower-resolution acquisition by applying spatial filters to the
original data in order to compare metrics, such as prediction accuracy, across
a range of spatial frequencies \citep[see][]{swisher_2010}.  However, this
approach has not gone unchallenged as it is unclear to what degree particular
filtering strategies \citep[e.g. Gaussian filtering vs. low-pass filtering in
the spatial frequency domain, see][]{misaki_2013} can effectively simulate the
properties of fMRI recorded at a lower physical resolution, where a change in
slice thickness alone can significantly alter image contrast. Despite this
criticism, we are not aware of any study that has compared the performance of
orientation decoding in visual cortex across a range of physical acquisition
resolutions.

In this study, we provide empirical data on the effect of spatial acquisition
resolution on the decoding of visual orientation from high field (\sevenT)
fMRI. We recorded BOLD fMRI data at \mm{0.8}, \mm{1.4}, \mm{2} and \mm{3} voxel
size while participants were visually stimulated with oriented phase-flickering
gratings using a uniform event-related paradigm.
% optimal resolution
These data enable estimation of the \textit{optimal acquisition resolution} for
orientation decoding from \sevenT\ fMRI.  \citet{chaimow_2011} investigated the
decoding of the stimulated hemifield using simulated \threeT\ fMRI data based
on a model of ocular dominance columns. They found that a resolution of \mm{3}
was optimal for decoding and performance decreased with higher or lower
resolution.  As it is known that in monkeys the organization of orientation
columns is characterized by higher spatial frequencies than ocular dominance
columns \citep{obermayer_1993}, we expect any optimal resolution for
orientation decoding to be higher than \mm{3}. Moreover, the BOLD point-spread
function (PSF) at \sevenT\ is known to be considerably smaller than at \threeT\
\citep[\mm{$\approx$2.3} FWHM vs.  \mm{$\approx$3.5} FWHM][]{shmuel_2007,
engel_1997} which should further increase the optimal resolution for decoding
at \sevenT.
% filtering
Multi-resolution data also allow for evaluating \textit{filtering strategies}
used in previous studies in terms of their validity regarding the simulation of
lower-resolution fMRI acquisitions from high-resolution data.
% aliasing
These data also enable the investigation of the \textit{role of aliasing} of a
high spatial-frequency signal (beyond the Nyquist frequency) into a
lower frequency range sampled by fMRI voxels \citep[sometimes referred to as
“hyperacuity”;][]{swisher_2010,opdebeeck_2010}, as, in the case of spatial
aliasing, the frequency bands carrying an orientation-selective signal would
vary with the sampling resolution of fMRI.
% veins
Lastly, we collected high-resolution susceptibility weighted imaging data for
blood-vessel localization in order to investigate the \textit{role of large
draining veins} that may carry orientation-selective signals reflected in low
spatial frequency components when sampled by millimeter range voxels
\citep{kamitani_2005,kriegeskorte_2007,shmuel_2010, gardner_2010}. In
combination with the multi-resolution fMRI data, we can investigate the effect
of this potential signal source on the orientation decoding at various levels
of spatial scale.



%% vim:foldmethod=expr
%% vim:fde=getline(v\:lnum)=~'^%%%%\ .\\+'?'>1'\:'='
%%% Local Variables: 
%%% mode: latex
%%% mode: auto-fill
%%% mode: flyspell
%%% eval: (ispell-change-dictionary "en_US")
%%% TeX-master: "main"
%%% End: 
