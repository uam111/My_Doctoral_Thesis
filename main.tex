%%%% Time-stamp: <2015-04-05 11:36:28 vk>
%% ========================================================================
%%%% Disclaimer
%% ========================================================================
%%
%% created by
%%
%%      Karl Voit
%%
%% modified by Michael Hanke for OvGU

%% ========================================================================
%%%% Basic settings
%% ========================================================================
%% (idea of using newcommands for basic documentclass settings from: Thomas Schlager)

\newcommand{\mypapersize}{A4}
%% e.g., "A4", "letter", "legal", "executive", ...
%% The size of the paper of the resulting PDF file.

\newcommand{\mylaterality}{twoside}
%% "oneside" or "twoside"
%% Either you are creating a document which is printed on both, left pages
%% and right pages (twoside) or you create a document which is printed
%% on right pages only (oneside).

\newcommand{\mydraft}{false}
%% "true" or "false"
%% Use draft mode? If true, included graphics are replaced by empty
%% rectangles (of same size) and overfull boxes (in margin space) are
%% marked with black box (-> easy to spot!)

\newcommand{\myparskip}{half}
%% e.g., "no", "full", "half", ...
%% How to separate paragraphs: indention ("no") or spacing ("half",
%% "full", ...).

\newcommand{\myBCOR}{0mm}
%% Inner binding correction. This value depends on the method which is
%% being used to bind your printed result. Some techniques do not
%% require a binding correction at all ("0mm"), other require for
%% example "5mm". Refer to KOMA script documentation for a detailed
%% explanation what a binding correction is and how to measure it.

\newcommand{\myfontsize}{12pt}
%% e.g., 10pt, 11pt, 12pt
%% The font size of the main text in pt (points).

\newcommand{\mylinespread}{2.0}
%% e.g., 1.0, 1.5, 2.0
%% Line spacing in %/100. For example 1.5 means 150% of the usual line
%% spacing. Please use with caution: 100% ("1.0") is fine because the
%% font was designed for it.

%% ========================================================================
%%%% Document metadata
%% ========================================================================

% First define the language:
%% "english,ngerman", "ngerman,english", ...
%% NOTE: The *last* language is the active one!
%% See babel documentation for further details.
%\newcommand{\mylanguage}{american,ngerman}
\newcommand{\mylanguage}{ngerman, american}

%% general metadata:
\newcommand{\myauthor}{Ayan Sengupta}  %% also used for PDF metadata (hyperref)
\newcommand{\myid}{206221} %% Matrikelnummer/Student ID
\newcommand{\mytitle}{TITLE}  %% also used for PDF metadata (hyperref)
\newcommand{\mysubtitle}{SUBTITLE}  %% also used for PDF metadata (hyperref)
\newcommand{\mysubject}{SUBJECT}  %% also used for PDF metadata (hyperref)
\newcommand{\mykeywords}{KEYWORDS}  %% also used for PDF metadata (hyperref)
\newcommand{\mysupervisor}{Dr.Stefan Pollmann} %% your supervisor
\newcommand{\mycosupervisor}{Dr.Michael Hanke} %% your supervisor
\newcommand{\mysubmissionmonth}{November} %% month you are handing in
\newcommand{\mysubmissionyear}{2015} %% year you are handing in
\newcommand{\mysubmissiontown}{Magdeburg} %% town of handing in

% 1st in English: German below (un)comment according to needs
%% this information is used only for generating the title page:
%\newcommand{\myworktitle}{Master's Thesis}  %% official type of work like ``Master theses''
%\newcommand{\mygrade}{Master of Science} %% title you are getting with this work like ``Master of ...''
%\newcommand{\mystudy}{Psychology} %% your study like ``Arts''
%\newcommand{\myuniversity}{Otto-von-Guericke-University} %% your university/school
%\newcommand{\myfaculty}{Fakulty of Natural Sciences} %% affiliation
%\newcommand{\myinstitute}{Institute of Psychology II} %% affiliation

%% 2nd German
%%% this information is used only for generating the title page:
\newcommand{\myworktitle}{Doctoral Thesis}  %% official type of work like ``Master theses''
\newcommand{\mygrade}{Doctor rerum naturalium} %% title you are getting with this work like ``Master of ...''
\newcommand{\mystudy}{Experimental Psychology} %% your study like ``Arts''
\newcommand{\myuniversity}{Otto-von-Guericke-Universit\"at} %% your university/school
\newcommand{\myfaculty}{Fakult\"at f\"ur Naturwissenschaften} %% affiliation
\newcommand{\myinstitute}{Institut f\"ur Psychologie II} %% affiliation
\newcommand{\purl}[2]{#1\footnote{\url{#2}}}

\newcommand{\sevenT}{\unit[7]{Tesla}}
\newcommand{\threeT}{\unit[3]{Tesla}}
\newcommand{\mm}[1]{\unit[#1]{mm}}
\newcommand{\seconds}[1]{\unit[#1]{s}}
\newcommand{\bandofinterest}{\unit[$\approx$5-8]{mm}}
%% Miscellaneous Latin abbreviations emphasizing
%% according to
%% http://en.wikipedia.org/wiki/List_of_Latin_phrases_(C-E)#endnote_egie
%% ``American style guides tend to recommend that "e.g."  and "i.e."
%% should generally be followed by a comma, just as "for example" and
%% "that is" would be; UK style tends to omit the comma''
\newcommand{\ie}[0]{\emph{i.e.},\ }
\newcommand{\eg}[0]{\emph{e.g.},\ }
\newcommand{\etc}[0]{\emph{etc.}}

%% Unified references
\newcommand{\fig}[1]{{Figure~\ref{fig:#1}}}

% Cross-referencing (comment out if forbidden)
\newcommand{\msec}[1]{Section~\ref{sec:#1}}
\newcommand{\mtab}[1]{Table~\ref{tab:#1}}
\newcommand{\msecs}[2]{Sections~\ref{sec:#1} and~\ref{sec:#2}}


%% BibLaTeX-settings: (see biblatex reference for further description)
\newcommand{\mybiblatexstyle}{authoryear}
%% e.g., "alphabetic", "authoryear", ...
%% The biblatex style which is being used for referencing. See
%% biblatex documentation for further details and more values.
%%
%% CAUTION: if you change the style, please check for (in)compatible
%%          "biblatex" package options in the file
%%          "template/preamble.tex"! For example: "alphabetic" does
%%          not have an option "dashed=..." and causes an error if it
%%          does not get removed from the list of options.

\newcommand{\mybiblatexdashed}{false}  %% "true" or "false"
%% If true: replace recurring reference authors with a dash.

\newcommand{\mybiblatexbackref}{true}  %% "true" or "false"
%% If true: create backward links from reference to citations.

\newcommand{\mybiblatexfile}{bib/references-biblatex.bib}
%% Name of the biblatex file that holds the references.

\newcommand{\mydispositioncolor}{0,0,0}
%% e.g., "30,103,182" (blue/turquois), "0,0,0" (black), ...
%% Color of the headings and so forth in RGB (red,green,blue) values.
%% NOTE: if you are using "0,0,0" for black, printers might still
%%       recognize pages as color pages. In case this is a problem
%%       (paying for color print-outs vs. paying for b/w-printouts)
%%       please edit file "template/preamble.tex" and change
%%       "\definecolor{DispositionColor}{RGB}{\mydispositioncolor}"
%%       to "\definecolor{DispositionColor}{gray}{0}" and thus
%%       overwriting the value of \mydispositioncolor above.

\newcommand{\mycolorlinks}{true}  %% "true" or "false"
%% Enables or disables colored links (hyperref package).

\newcommand{\mytitlepage}{template/title_thesis_ovgu_psych}
%% Your own or one of following pre-defined title pages:
%% "template/title_plain_maketitle": simple maketitle page
%% "template/title_Diplomarbeit_KF_Uni_Graz.tex": fancy (german) title page for KF Uni Graz
%% "template/title_Thesis_TU_Graz":
%%             titlepage for Graz University of Technology (correct
%%             (old?) Corporate Design) by Karl Voit (2012)
%% "template/title_Thesis_TU_Graz_-_kazemakase":
%%             titlepage for Graz University of Technology
%%             (correct new Corporate Design) by kazemakase (2013):
%%             see https://github.com/novoid/LaTeX-KOMA-template/issues/5
%% "template/title_VWA": titlepage for Vorwissenschaftliche Arbeit

\newcommand{\mytodonotesoptions}{}
%% e.g., "" (empty), "disable", ...
%% Options for the todonotes-package. If "disable", all todonotes will
%% be hidden (including listoftodos).

%% Load main settings for document preamble:
\input{template/preamble}%% DO NOT REMOVE THIS LINE!

\setboolean{myaddcolophon}{false}  %% "true" or "false"
%% If set to "true": a colophon (with notes about this document
%% template, LaTeX, ...) is added after the title page.
%% Please do not set to "false" without a good reason. The colophon
%% helps your readers to get in touch with LaTeX and to find this template.

\setboolean{myaddlistoftodos}{true}  %% "true" or "false"
%% If set to "true": the current list of open todos is added after the
%% table of contents. If \mytodonotesoptions is set to "disable", no
%% list of todos is added, independent of this setting here.




%% ========================================================================
%%%% MISC command definitions
%% ========================================================================
\input{template/mycommands}

%% ========================================================================
%%%% Typographic settings
%% ========================================================================
\input{template/typographic_settings}


%% ========================================================================
%%%% MISC usepackages
%% ========================================================================

%% ... it's OK to put here your own usepackage commands ...




%% ========================================================================
%%%% MISC self-defined commands and settings
%% ========================================================================

%% ... it's OK to put here your own newcommand/newenvironment-definitions ...




\newcommand{\myLaT}{\LaTeX{}@TUG\xspace} %% LaTeX@TUG text "logo"

\hyphenation{ex-am-ple hy-phen-ate}  %% in order to use German umlauts
%% here (Ver-\"of-fent-li-chung), you have to check for
%% activated \usepackage[T1]{fontenc} in the preamble

%% override default language of babel: (be sure to know, what you're
%% doing here)
%\selectlanguage{american}
%\selectlanguage{ngerman}

%% ========================================================================
%%%% Templates
%% ========================================================================

%% template for inserting figures:
% \myfig{}%% filename
%       {}%% width/height
%       {}%% caption
%       {}%% optional (short) caption for list of figures
%       {fig:}%% label

%% acronyms in small caps: \myacro{UNESCO}


\input{template/pdf_settings}  %% should be *last* definitions in preamble!
%% ========================================================================
%%%% begin{document}
%% ========================================================================
\begin{document}

\frontmatter                    %% KOMA: roman page numbers and such; only available in scrbook

\input{template/colophon}                %% defines information about editor, LaTeX, font, ...

%% Choose your desired title page:
\input{\mytitlepage}            %% include title page

%% if myaddlistoftodos is set to "true", the current list of open todos is added:
\ifthenelse{\boolean{myaddlistoftodos}}{
  \newpage\listoftodos\newpage  %% handy if you are using todonotes with \todo{}
}{}                             %% with todonotes-package option "disable" you can get rid of any todo in the output

\input{template/declaration_ovgu}  %% Statutory Declaration

%% include the abstract without chapter number but include it on table of contents:
\cleardoublepage
\include{abstract}              %% Abstract

%% 
%% This is file `thesis.cls', for use at Rensselaer Polytechnic Institute.
%% It is based on the standard LaTeX report class.
%% Originally written by Harriet Borton (ITS), April 1996.
%% Modified April 11, 1996; minor revisions February, 2001.
%% Modified Oct 5, 2006; better compatibility with hyperref package.
%% Modified Dec 17 2007; change title page for MS degree as requested by OGS.
%% Modified Jan 28 2008; provide abstract title page for MS, incorporate 3-advisers.
%% Modified Jan 6 2009; small change to title page to avoid warning w/hyperref pkg
%% 
%% On RCS, template files for preparing a thesis are in the directory
%% /dept/arc/docs/latex-thesis/
%% 
%% This file is part of the LaTeX2e system. 
%% ---------------------------------------- 
%% 
\NeedsTeXFormat{LaTeX2e}
\ProvidesClass{thesis}[2009/01/06 Rensselaer Polytechnic Institute]
%  Note that the setspace package is built in (code included near the end of 
%  this file) to provide "line-and-a half spacing" (1.4 by default) and also
%  the singlespace environment. 

% RPI option chap:
\newif\ifchap  % true for chap option
  \chapfalse   % false by default
\DeclareOption{chap}{\chaptrue} % option to print "Chapter" at each new chapter
\newcommand\docsize{}  % to allow 10pt or 11pt to be specified as option
\DeclareOption{10pt}{\renewcommand\docsize{10pt}}
\DeclareOption{11pt}{\renewcommand\docsize{11pt}}
\DeclareOption{12pt}{\renewcommand\docsize{12pt}}
%  Prepare to load the standard report class (12pt):
\DeclareOption*{\PassOptionsToClass{\CurrentOption}{report}}
\ExecuteOptions{12pt}         % define 12pt as the default doc size
\ProcessOptions
\LoadClass[\docsize]{report}  % load report.cls
%%%%%%%%%%%%%%%%%%%%%%%%%%%%%%%%%%%%%%%%%%%%%%%%%%%%%%%%%%%%%%%%%%%%%%%%%%%%


%  The following sections are revisions or additions to report.cls
%
%%%%%%%%%%%%%%%%%%%%%%%%%%%%%%%%%%%%%%%%%%%%%%%%%%%%%%%%%%%%%%%%%%%%%%%%%%%%
%  FOOTNOTES:   make them continuously numbered throughout document
%%%%%%%%%%%%%%%%%%%%%%%%%%%%%%%%%%%%%%%%%%%%%%%%%%%%%%%%%%%%%%%%%%%%%%%%%%%%
% define command that can undo having footnotes reset with each chapter:
% (taken from removefr.sty by Donald Arseneau) 
\def\@removefromreset#1#2{\let\@tempb\@elt
   \expandafter\let\expandafter\@tempa\csname c@#1\endcsname
   \def\@elt##1{\expandafter\ifx\csname c@##1\endcsname\@tempa\else
         \noexpand\@elt{##1}\fi}%
   \expandafter\edef\csname cl@#2\endcsname{\csname cl@#2\endcsname}%
   \let\@elt\@tempb}
% use the command \@removefromreset to undo the \@addtoreset in report.cls:
\@removefromreset{footnote}{chapter}
%
% define command to allow people to reset footnote counter at will: 
\def\resetfootnote{\setcounter{footnote}{0}}  % definition to reset footnote


%%%%%%%%%%%%%%%%%%%%%%%%%%%%%%%%%%%%%%%%%%%%%%%%%%%%%%%%%%%%%%%%%%%%%%%%%
% PAGE LAYOUT
%%%%%%%%%%%%%%%%%%%%%%%%%%%%%%%%%%%%%%%%%%%%%%%%%%%%%%%%%%%%%%%%%%%%%%%%%
% SIDE MARGINS:
\if@twoside                 % Values for two-sided printing:
   \oddsidemargin .55in     %   Left margin on odd-numbered pages.
   \evensidemargin .05in    %   Left margin on even-numbered pages.
   \marginparwidth 40pt     %   Width of marginal notes.
\else                       % Values for one-sided printing:
   \oddsidemargin 0.55in    %   Note that \oddsidemargin = \evensidemargin
   \evensidemargin 0.55in
   \marginparwidth 40pt
\fi
\marginparsep 10pt          % Horizontal space between outer margin and
                            % marginal note
\textwidth 5.9in            % width of text
 
% VERTICAL SPACING:
                         % Top of page:
\topmargin -.5in         %    distance from top of page to running head
\headheight 14pt         %    Height of box containing running head.
\headsep .4in            %    Space between running head and text.
\textheight 8.8in        %    space for text
\footskip 30pt           %    Distance from baseline of box containing foot
                         %    to baseline of last line of text.


%%%%%%%%%%%%%%%%%%%%%%%%%%%%%%%%%%%%%%%%%%%%%%%%%%%%%%%%%%%%%%%%%%%%%%%%%%%%
%                            SECTION HEADINGS
%%%%%%%%%%%%%%%%%%%%%%%%%%%%%%%%%%%%%%%%%%%%%%%%%%%%%%%%%%%%%%%%%%%%%%%%%%%% 
\newcommand\chaptersize{\large}
\newcommand\sectionsize{\large}
\newcommand\subsectionsize{\normalsize}
\newcommand\subsubsectionsize{\normalsize}
\newcounter{firstchapter}
\setcounter{firstchapter}{0}

\setcounter{secnumdepth}{3}    % Number subsubsections in the chapters
\setcounter{tocdepth}{3}       % Put subsubsections in the table of contents

% Print "CHAPTER" if chap option is specified:
\ifchap
  \renewcommand\@chapapp{\chaptername}
\else
  \renewcommand\@chapapp{}
\fi

\def\specialhead#1{%   GENERAL HEADING WITHOUT A NUMBER (for abstract, etc.)
     \ifx\phantomsection\undefined
     \else
        \clearpage\phantomsection
     \fi
     \addcontentsline{toc}{chapter}{#1}
     \chapter*{\centering #1 \@mkboth{#1}{#1}}}


\def\@chapter[#1]#2{\ifnum\c@firstchapter=0    % start of rpi added stuff
                      \if@twoside\cleardoublepage\suppressfloats[t]\fi
                      \pagenumbering{arabic} 
                      \setcounter{firstchapter}{1}
                    \fi
                    \renewcommand{\thepage}{\arabic{page}}
                    \thispagestyle{plain}
                    \pagestyle{myheadings}      % end of rpi added stuff
                    \ifnum \c@secnumdepth >\m@ne
                         \refstepcounter{chapter}%
                         \typeout{\@chapapp\space\thechapter.}%
                         \addcontentsline{toc}{chapter}%
                                   {\protect\numberline{\thechapter.}#1}%
                    \else
                      \addcontentsline{toc}{chapter}{#1}%
                    \fi
                    \chaptermark{#1}% 
%                    \addtocontents{lof}{\protect\addvspace{10\p@}}%
%                    \addtocontents{lot}{\protect\addvspace{10\p@}}%
                    \if@twocolumn
                      \@topnewpage[\@makechapterhead{#2}]%
                    \else
                      \@makechapterhead{#2}%
                      \@afterheading
                    \fi}
\def\@makechapterhead#1{%
  \vspace*{0\p@}%
  {\parindent \z@ \raggedright \centering \normalfont \chaptersize
    \ifnum \c@secnumdepth >\m@ne
      \ifchap
         \bfseries \@chapapp{} \thechapter    % print "Chapter" and number
         \vskip -3pt           %\par\nobreak (original)
       \else
         \bfseries \thechapter. 
       \fi
    \fi
    \interlinepenalty\@M
    \bfseries #1\par\nobreak
    \vskip 15\p@
  }}

\def\@makeschapterhead#1{%    heading for chapter* command (no numbering)
  \vspace*{0\p@}%
  {\parindent \z@ \raggedright \centering
    \normalfont  \chaptersize 
    \interlinepenalty\@M
    \bfseries  #1\par\nobreak
    \vskip 15\p@
  }}

\renewcommand\section{\@startsection {section}{1}{\z@}%
                                   {3.5ex \@plus 1ex \@minus .2ex}%
                                   {.5ex \@plus .3ex}%{1.4ex \@plus.2ex}%
                                   {\normalfont\sectionsize\bfseries}}
\renewcommand\subsection{\@startsection{subsection}{2}{\z@}%
                                     {3.25ex\@plus 1ex \@minus .2ex}%
                                     {.3ex \@plus .2ex}%{1.2ex \@plus .2ex}%
                                     {\normalfont\subsectionsize\bfseries}}
\renewcommand\subsubsection{\@startsection{subsubsection}{3}{\z@}%
                                     {3.25ex\@plus 1ex \@minus .2ex}%
                                     {.2ex \@plus .1ex}%{1ex \@plus .2ex}%
                                     {\normalfont\subsubsectionsize\bfseries}}
% \paragraph and \subparagraph headings unchanged from report.cls.


%%%%%%%%%%%%%%%%%%%%%%%%%%%%%%%%%%%%%%%%%%%%%%%%%%%%%%%%%%%%%%%%%%%%%%%%%%%%
%  APPENDIX
%%%%%%%%%%%%%%%%%%%%%%%%%%%%%%%%%%%%%%%%%%%%%%%%%%%%%%%%%%%%%%%%%%%%%%%%%%%%
\renewcommand\appendix{\par
  \setcounter{chapter}{0}%
  \setcounter{section}{0}%
  \chaptrue
  \renewcommand\@chapapp{\appendixname}%
  \renewcommand\thechapter{\@Alph\c@chapter}}


%%%%%%%%%%%%%%%%%%%%%%%%%%%%%%%%%%%%%%%%%%%%%%%%%%%%%%%%%%%%%%%%%%%%%%%%%%%%
%  FIGURES and TABLES
%%%%%%%%%%%%%%%%%%%%%%%%%%%%%%%%%%%%%%%%%%%%%%%%%%%%%%%%%%%%%%%%%%%%%%%%%%%%
%control float placement:
%
\setcounter{topnumber}{2}
\renewcommand\topfraction{.8}
\setcounter{bottomnumber}{2}
\renewcommand\bottomfraction{.8}
\setcounter{totalnumber}{4}
\renewcommand\textfraction{.2}
\renewcommand\floatpagefraction{.8}
\setcounter{dbltopnumber}{2}
\renewcommand\dbltopfraction{.8}
\renewcommand\dblfloatpagefraction{.8}

\long\def\@makecaption#1#2{%
  \vskip\abovecaptionskip
  \sbox\@tempboxa{\bfseries#1:~~#2}%   Make caption bold
  \ifdim \wd\@tempboxa >\hsize
   {\bfseries #1:~~#2}\par%            Make caption bold
  \else
    \global \@minipagefalse
    \hb@xt@\hsize{\hfil\box\@tempboxa\hfil}%
  \fi
  \vskip\belowcaptionskip}

%%%%%%%%%%%%%%%%%%%%%%%%%%%%%%%%%%%%%%%%%%%%%%%%%%%%%%%%%%%%%%%%%%%%%%%%%%%%
%  TABLE of CONTENTS,  LIST OF TABLES,  LIST OF FIGURES 
%%%%%%%%%%%%%%%%%%%%%%%%%%%%%%%%%%%%%%%%%%%%%%%%%%%%%%%%%%%%%%%%%%%%%%%%%%%%

\renewcommand\tableofcontents{%
    \if@twocolumn
      \@restonecoltrue\onecolumn
    \else
      \@restonecolfalse
    \fi
    \chapter*{\contentsname
        \@mkboth{%
           \MakeUppercase\contentsname}{\MakeUppercase\contentsname}}%
    \vskip -1em \begin{singlespace}      % singlespacing
    \@starttoc{toc}%
    \if@restonecol\twocolumn\fi
    \end{singlespace}
    }
 
\renewcommand*\l@chapter{\pagebreak[3]\vskip 10pt plus 1pt minus 1pt
                         \@dottedtocline{0}{0em}{1.4em}}
\renewcommand*\l@section{\vskip 6pt plus 1pt minus 1pt
                         \@dottedtocline{1}{1.5em}{2.3em}}
\renewcommand*\l@subsection{\ifnum\c@tocdepth>1\vskip 4pt minus 1pt \fi
                         \@dottedtocline{2}{3.8em}{3.2em}}
\renewcommand*\l@subsubsection{\ifnum\c@tocdepth>2 \vskip 3pt minus 1pt \fi
                         \@dottedtocline{3}{7.0em}{4.1em}}

% The following removed because it's not consistent with entries from longtable
%% modify the definition below (taken from latex.ltx) to include 
%% "Table" and "Figure" in entries for lot and lof: 
%\long\def\@caption#1[#2]#3{%\baselineskip 14.5 pt
%  \addcontentsline{\csname ext@#1\endcsname}{#1}%
%  {\protect\numberline{\csname fnum@#1\endcsname}{\ignorespaces #2}}%
%  \begingroup
%    \@parboxrestore
%    \normalsize
%    \@makecaption{\csname fnum@#1\endcsname}{\ignorespaces #3}\par
%  \endgroup}
 
\renewcommand\listoffigures{%
    \if@twocolumn
      \@restonecoltrue\onecolumn
    \else
      \@restonecolfalse
    \fi
    \ifx\phantomsection\undefined
      \else
        \clearpage\phantomsection
      \fi
   \chapter*{\centering\listfigurename  % center it
      \@mkboth{\MakeUppercase\listfigurename}%
              {\MakeUppercase\listfigurename}}%
    \addcontentsline{toc}{chapter}{\listfigurename}  % add lof to toc
    \vskip -1em \begin{singlespace}  % singlespacing
    \@starttoc{lof}%
    \if@restonecol\twocolumn\fi
    \end{singlespace}
    }

\renewcommand\listoftables{%
    \if@twocolumn
      \@restonecoltrue\onecolumn
    \else
      \@restonecolfalse
    \fi
    \ifx\phantomsection\undefined
      \else
        \clearpage\phantomsection
      \fi
    \chapter*{\centering\listtablename  % center it
      \@mkboth{\MakeUppercase\listtablename}%
              {\MakeUppercase\listtablename}}%
\addcontentsline{toc}{chapter}{\listtablename}  % add lot to toc
        \vskip -1em \begin{singlespace}  % singlespacing
    \@starttoc{lot}%
    \if@restonecol\twocolumn\fi
    \end{singlespace}
    }

%remove following at same time as remove code to put "Figure" in LOF
%\renewcommand*\l@figure{\vskip 10pt plus 1pt minus 1pt
%                       \@dottedtocline{1}{0em}{5.8em}}
 
\renewcommand*\l@figure{\vskip 10pt plus 1pt minus 1pt
                       \@dottedtocline{1}{0em}{2.8em}}
\let\l@table\l@figure

\let\rpicaption\caption
\let\lrpicaption\caption


%%%%%%%%%%%%%%%%%%%%%%%%%%%%%%%%%%%%%%%%%%%%%%%%%%%%%%%%%%%%%%%%%%%%%%
%  BIBLIOGRAPHY
%%%%%%%%%%%%%%%%%%%%%%%%%%%%%%%%%%%%%%%%%%%%%%%%%%%%%%%%%%%%%%%%%%%%%%
% RPI def: for use in making an unnumbered bibliography with hanging indents
\def\bibentry{\vskip10pt\par\noindent\hangindent=40pt\frenchspacing}

% report definition modified for no automatic heading and use ragged right
\newcommand{\bibalign}{\raggedright}

\renewenvironment{thebibliography}[1]
     {\bibalign\frenchspacing
      \list{\@biblabel{\@arabic\c@enumiv}}%
           {\settowidth\labelwidth{\@biblabel{#1}}%
            \leftmargin\labelwidth
            \advance\leftmargin\labelsep
            \@openbib@code
            \usecounter{enumiv}%
            \let\p@enumiv\@empty
            \renewcommand\theenumiv{\@arabic\c@enumiv}}%
      \sloppy\clubpenalty4000\widowpenalty4000%
      \sfcode`\.=\@m}
     {\def\@noitemerr
       {\@latex@warning{Empty `thebibliography' environment}}%
      \endlist}


%%%%%%%%%%%%%%%%%%%%%%%%%%%%%%%%%%%%%%%%%%%%%%%%%%%%%%%%%%%%%%%%%%%%%
%  FOR RPI TITLEPAGE, ABSTR. TITLEPAGE & COPYRIGHT PAGE 
%%%%%%%%%%%%%%%%%%%%%%%%%%%%%%%%%%%%%%%%%%%%%%%%%%%%%%%%%%%%%%%%%%%%%

 \def\thesistitle#1{\gdef\@thesistitle{#1}}
 \def\author#1{\gdef\@author{#1}}
 \def\degree#1{\gdef\@degree{#1}}
 \def\department#1{\gdef\@department{#1}}
 \newcount\numcomm \numcomm=1
 \def\signaturelines#1{\numcomm=#1}
 \def\thadviser#1{\gdef\@thadviser{#1}} 
 \def\projadviser#1{\gdef\@projadviser{#1}} \projadviser{*}
 \def\cothadviser#1{\gdef\@cothadviser{#1}} \cothadviser{*} 
 \def\coprojadviser#1{\gdef\@coprojadviser{#1}} \coprojadviser{*}
 \def\cocothadviser#1{\gdef\@cocothadviser{#1}} \cocothadviser{*}
 \def\cocoprojadviser#1{\gdef\@cocoprojadviser{#1}} \cocoprojadviser{*} 
 \def\doctype{\if \@projadviser *Dissertation \else Thesis \fi}
 \def\adviser{\if \@projadviser *\@thadviser \else \@projadviser\fi}
 \def\coadviser{\if \@coprojadviser *\@cothadviser\else \@coprojadviser\fi}
 \def\cocoadviser{\if \@cocoprojadviser *\@cocothadviser\else \@cocoprojadviser\fi}
 \def\memberone#1{\gdef\@memberone{#1}}
 \def\membertwo#1{\gdef\@membertwo{#1}}
 \def\memberthree#1{\gdef\@memberthree{#1}}
 \def\memberfour#1{\gdef\@memberfour{#1}}
 \def\memberfive#1{\gdef\@memberfive{#1}}
 \def\membersix#1{\gdef\@membersix{#1}}
 \def\copyrightyear#1{\gdef\@copyrightyear{#1}} \copyrightyear{\the\year}
 \def\submitdate#1{\gdef\@submitdate{#1}}

% Format signature lines on title page
\newcount\numcount
\def\sigblock{
      \numcount=\numcomm
      %\leftline{Approved by the\hfil}
      \leftline{Examining Committee:\hfil}
      \vskip 28pt \vfil 
      \hrule width 2.8in \hfil \vskip -3pt
      \leftline{\adviser, \doctype Adviser\hfil} 
      \advance \numcount by -1
      \if \coadviser * \else % there's a co-adviser
        \vskip 19pt\vfil\hrule width 2.8in\hfil\vskip -3pt
        \leftline{\coadviser, \doctype Adviser\hfil}
        \advance \numcount by -1
      \fi
      \ifnum\numcount > 0 %
      \vskip 19pt\vfil\hrule width 2.8in\hfil\vskip -3pt
      \leftline{\@memberone, Member\hfil} \fi %
      \advance \numcount by -1 \ifnum\numcount > 0 %
      \vskip 19pt\vfil\hrule width 2.8in\hfil\vskip -3pt
      \leftline{\@membertwo, Member\hfil} \fi %
      \advance \numcount by -1 \ifnum\numcount > 0
      \vskip 19pt\vfil\hrule width 2.8in\hfil\vskip -3pt
      \leftline{\@memberthree, Member\hfil}  \fi
      \advance \numcount by -1 \ifnum\numcount > 0
      \vskip 19pt\vfil\hrule width 2.8in\hfil\vskip -3pt
      \leftline{\@memberfour, Member\hfil}   \fi
      \advance \numcount by -1 \ifnum\numcount > 0
      \vskip 19pt\vfil\hrule width 2.8in\hfil\vskip -3pt
      \leftline{\@memberfive, Member\hfil}  \fi
      \advance \numcount by -1 \ifnum\numcount > 0
      \vskip 19pt\vfil\hrule width 2.8in\hfil\vskip -3pt
      \leftline{\@membersix, Member\hfil} \fi
} 

 \def\titlepage{%
     \pagenumbering{roman}
     \thispagestyle{empty}
     \ifnum\numcomm<4 \vglue .5in\vfil \fi  % masters
     \ifnum\numcomm>5 \hbox{ } \vspace{-24pt}\fi % make more space on page
     \begin{singlespace}
     \begin{center}
         \parindent=0pt
         {\large\uppercase\expandafter{\@thesistitle}}\\ [12pt]
         By \\ [12pt]
         \@author\\ [12pt]
         A \doctype Submitted to the Graduate\\ [8pt]
         Faculty of Rensselaer Polytechnic Institute\\ [8pt]
         in Partial Fulfillment of the\\ [8pt]
         Requirements for the Degree of\\ [8pt]
         \uppercase\expandafter{\@degree}\\ [8pt]
         Major Subject:~~\uppercase\expandafter{\@department}\\ 
         \ifnum\numcomm < 7 \vskip 32pt \else \vskip 24pt \fi
         \sigblock 
         \ifnum \numcomm < 7 \vskip 32pt \else \vskip 24pt \fi
         Rensselaer Polytechnic Institute\\
         Troy, New York\\ [14pt]
         \ifnum\numcomm>6 \vskip -8pt \fi
         \@submitdate
     \end{center}
     \end{singlespace}
     \newpage
     \pagestyle{plain}
%    \pagenumbering{roman}
%    \setcounter{page}{2}
}
 
\def\copyrightpage{
    \hbox{ }
    \vfill
    \begin{center}
    \copyright\ Copyright \@copyrightyear \\
    by \\
    \@author \\
    All Rights Reserved \\ [12pt]
    \end{center}
    \clearpage}


%%%%%%%%%%%%%%%%%%%%%%%%%%%%%%%%%%%%%%%%%%%%%%%%%%%%%%%%%%%%%%%%%%%%%
% The CODE FROM SETSPACE.STY:
%%%%%%%%%%%%%%%%%%%%%%%%%%%%%%%%%%%%%%%%%%%%%%%%%%%%%%%%%%%%%%%%%%%%%
\def\setspace@size{%
  \ifx \@currsize \normalsize
    \@normalsize
  \else
    \@currsize
  \fi
}

\def\doublespacing{%
  \ifcase \@ptsize \relax % 10pt
    \def \baselinestretch {1.667}%
  \or % 11pt
    \def \baselinestretch {1.618}%
  \or % 12pt
    \def \baselinestretch {1.655}%
  \fi
  \setspace@size
}

\def\onehalfspacing{%
  \ifcase \@ptsize \relax % 10pt
    \def \baselinestretch {1.25}%
  \or % 11pt
    \def \baselinestretch {1.213}%
  \or % 12pt
    \def \baselinestretch {1.241}%
  \fi
  \setspace@size
}

\def\singlespacing{%
  \def \baselinestretch {1}%
  \setspace@size
  \vskip \baselineskip  % Correction for coming into singlespace
}

\def\setstretch#1{%
  \renewcommand{\baselinestretch}{#1}}

%---Stretch the baseline BEFORE calculating the strut size. This improves
%   spacing below tabular environments etc., probably...
%   Comments are welcomed.

\def\@setsize#1#2#3#4{%
  % Modified 1993.04.07--GDG per KPC
  \@nomath#1%
  \let\@currsize#1%
  \baselineskip #2%
  \baselineskip \baselinestretch\baselineskip
  \parskip \baselinestretch\parskip
  \setbox\strutbox \hbox{%
    \vrule height.7\baselineskip depth.3\baselineskip width\z@}%
  \skip\footins \baselinestretch\skip\footins
  \normalbaselineskip\baselineskip#3#4}

%---Increase the space between last line of text and footnote rule.
%\skip\footins 20pt plus4pt minus4pt

%---Reset baselinestretch within floats and footnotes.

% GT:  This is where the conflict with the combination of the color
% package and the figure environment used to occur.

\let\latex@xfloat=\@xfloat
\def\@xfloat #1[#2]{%
  \latex@xfloat #1[#2]%
  \def \baselinestretch {1}\@normalsize  %%  GT: need \@normalsize?
  \normalsize
}

\long\def\@footnotetext#1{%
  \insert\footins{%
    \def\baselinestretch {1}%
    \reset@font\footnotesize
    \interlinepenalty\interfootnotelinepenalty
    \splittopskip\footnotesep
    \splitmaxdepth \dp\strutbox \floatingpenalty \@MM
    \hsize\columnwidth
    \@parboxrestore
    \protected@edef\@currentlabel{%
      \csname p@footnote\endcsname\@thefnmark}%
    \color@begingroup
      \@makefntext{%
        \rule\z@\footnotesep\ignorespaces#1\@finalstrut\strutbox}%
    \color@endgroup}}

\long\def\@mpfootnotetext#1{%
  \global\setbox\@mpfootins\vbox{%
    \unvbox \@mpfootins
    \def\baselinestretch {1}%
    \reset@font\footnotesize
    \hsize\columnwidth
    \@parboxrestore
    \protected@edef\@currentlabel{%
      \csname p@mpfootnote\endcsname\@thefnmark}%
    \color@begingroup
      \@makefntext{%
       \rule\z@\footnotesep\ignorespaces#1\@finalstrut\strutbox}%
   \color@endgroup}}

% A single spaced quote (say) is done by surrounding singlespace with quote.
%
\def\singlespace{%
  \vskip \baselineskip
  \def\baselinestretch {1}%
  \setspace@size
  \vskip -\baselineskip
}

\def\endsinglespace{\par}

%  spacing, doublespace and onehalfspace all are meant to INCREASE the
%  spacing (i.e. calling onehalfspace from within doublespace will not
%  produce a graceful transition between spacings)
%
% Next two definitions fixed for consistency with TeX 3.x

\def\spacing#1{%
  \par
  \begingroup             % moved from \endspacing by PGBR 29-1-91
    \def\baselinestretch {#1}%
    \setspace@size
}

\def\endspacing{%
    \par
    \vskip \parskip
    \vskip \baselineskip
  \endgroup
  \vskip -\parskip
  \vskip -\baselineskip
}

% one and a half spacing is 1.5 x pt size
\def\onehalfspace{%
  \ifcase \@ptsize \relax  % 10pt
    \spacing{1.25}%
  \or % 11pt
    \spacing{1.213}%
  \or % 12pt
    \spacing{1.241}%
  \fi
}

\let\endonehalfspace=\endspacing

% double spacing is 2 x pt size
\def\doublespace{%
  \ifcase \@ptsize \relax % 10pt
    \spacing{1.667}%
  \or % 11pt
    \spacing{1.618}%
  \or % 12pt
    \spacing{1.655}%
  \fi
}

\let\enddoublespace=\endspacing

% GT:  EMSH chose to omit display math part that follows.
% She wrote (see above) that the "altered spacing before and after displayed
% equations ... just looked too much".
%
% Fix up spacing before and after displayed math
% (arraystretch seems to do a fine job for inside LaTeX displayed math,
% since array and eqnarray seem to be affected as expected).
% Changing \baselinestretch and doing a font change also works if done here,
% but then you have to change @setsize to remove the call to @nomath)
%
\everydisplay\expandafter{%
  \the\everydisplay
  \abovedisplayskip \baselinestretch\abovedisplayskip
  \belowdisplayskip \abovedisplayskip
  \abovedisplayshortskip \baselinestretch\abovedisplayshortskip
  \belowdisplayshortskip \baselinestretch\belowdisplayshortskip
}
%%%%%%%%%%%%%%%%%%%%%%  End of setspace.sty %%%%%%%%%%%%%%%%%%%%%%%%%%%%



%%%%%%%%%%%%%%%%%%%%%%%%%%%%%%%%%%%%%%%%%%%%%%%%%%%%%%%%%%%%%%%%%%%%%
% SOME INITIALIZATIONS:
%%%%%%%%%%%%%%%%%%%%%%%%%%%%%%%%%%%%%%%%%%%%%%%%%%%%%%%%%%%%%%%%%%%%%
% make the following names uppercase:
\renewcommand\contentsname{CONTENTS}
\renewcommand\listfigurename{LIST OF FIGURES}
\renewcommand\listtablename{LIST OF TABLES}
\renewcommand\bibname{BIBLIOGRAPHY}
\renewcommand\indexname{INDEX}
\renewcommand\partname{PART}
\renewcommand\chaptername{CHAPTER}
\renewcommand\appendixname{APPENDIX}
\renewcommand\abstractname{ABSTRACT}

\renewcommand\chaptersize{\large}
\renewcommand\sectionsize{\large}
\renewcommand\subsectionsize{\normalsize}
\renewcommand\subsubsectionsize{\normalsize}

\raggedbottom

\setstretch{1.4}        % at 11 or 12pt this is line-and-a-half spacing
\parindent .4in         % Width of paragraph indentation

\markboth{}{}           % Do not include chapter titles in headers;
\pagestyle{myheadings}  % include just page numbers in upper right

\renewcommand{\bibalign}{\raggedright}  %  bibliography is ragged right
% for fully justified bibliography: \renewcommand{\bibalign}{}

\endinput
%% 
%% End of file `thesis.cls'.


%%%% end{document}
\end{document}
%% vim:foldmethod=expr
%% vim:fde=getline(v\:lnum)=~'^%%%%\ .\\+'?'>1'\:'='
%%% Local Variables:
%%% mode: latex
%%% mode: auto-fill
%%% mode: flyspell
%%% eval: (ispell-change-dictionary "en_US")
%%% TeX-master: "main"
%%% End:
