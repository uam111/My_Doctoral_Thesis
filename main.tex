%%%% Time-stamp: <2015-04-05 11:36:28 vk>
%% ========================================================================
%%%% Disclaimer
%% ========================================================================
%%
%% created by
%%
%%      Karl Voit
%%
%% modified by Michael Hanke for OvGU

%% ========================================================================
%%%% Basic settings
%% ========================================================================
%% (idea of using newcommands for basic documentclass settings from: Thomas Schlager)

\newcommand{\mypapersize}{A4}
%% e.g., "A4", "letter", "legal", "executive", ...
%% The size of the paper of the resulting PDF file.

\newcommand{\mylaterality}{twoside}
%% "oneside" or "twoside"
%% Either you are creating a document which is printed on both, left pages
%% and right pages (twoside) or you create a document which is printed
%% on right pages only (oneside).

\newcommand{\mydraft}{false}
%% "true" or "false"
%% Use draft mode? If true, included graphics are replaced by empty
%% rectangles (of same size) and overfull boxes (in margin space) are
%% marked with black box (-> easy to spot!)

\newcommand{\myparskip}{half}
%% e.g., "no", "full", "half", ...
%% How to separate paragraphs: indention ("no") or spacing ("half",
%% "full", ...).

\newcommand{\myBCOR}{0mm}
%% Inner binding correction. This value depends on the method which is
%% being used to bind your printed result. Some techniques do not
%% require a binding correction at all ("0mm"), other require for
%% example "5mm". Refer to KOMA script documentation for a detailed
%% explanation what a binding correction is and how to measure it.

\newcommand{\myfontsize}{12pt}
%% e.g., 10pt, 11pt, 12pt
%% The font size of the main text in pt (points).

\newcommand{\mylinespread}{2.0}
%% e.g., 1.0, 1.5, 2.0
%% Line spacing in %/100. For example 1.5 means 150% of the usual line
%% spacing. Please use with caution: 100% ("1.0") is fine because the
%% font was designed for it.

%% ========================================================================
%%%% Document metadata
%% ========================================================================

% First define the language:
%% "english,ngerman", "ngerman,english", ...
%% NOTE: The *last* language is the active one!
%% See babel documentation for further details.
%\newcommand{\mylanguage}{american,ngerman}
\newcommand{\mylanguage}{ngerman,american}

% 1st in English: German below (un)comment according to needs
%% general metadata:
\newcommand{\myauthor}{NAME SURNAME}  %% also used for PDF metadata (hyperref)
\newcommand{\mytitle}{TITLE}  %% also used for PDF metadata (hyperref)
\newcommand{\mysubtitle}{SUBTITLE}  %% also used for PDF metadata (hyperref)
\newcommand{\mysubject}{SUBJECT}  %% also used for PDF metadata (hyperref)
\newcommand{\mykeywords}{KEYWORDS}  %% also used for PDF metadata (hyperref)

%% this information is used only for generating the title page:
\newcommand{\myworktitle}{Master's Thesis}  %% official type of work like ``Master theses''
\newcommand{\mygrade}{Master of Science} %% title you are getting with this work like ``Master of ...''
\newcommand{\mystudy}{Telematik} %% your study like ``Arts''
\newcommand{\myuniversity}{Graz University of Technology} %% your university/school
\newcommand{\myfaculty}{Fakult\"at f\"ur Naturwissenschaften} %% affiliation
\newcommand{\myinstitute}{Institute for Softwaretechnology} %% affiliation
\newcommand{\mysupervisor}{Dr.~Some Body} %% your supervisor
\newcommand{\mycosupervisor}{Dr.~Some Body Else} %% your supervisor
\newcommand{\mysubmissionmonth}{November} %% month you are handing in
\newcommand{\mysubmissionyear}{2013} %% year you are handing in
\newcommand{\mysubmissiontown}{Magdeburg} %% town of handing in

%% additional information for generic_documentation title page
\newcommand{\myid}{1234567} %% Matrikelnummer


%% 2nd German
%\newcommand{\myauthor}{VORNAME NAME}  %% also used for PDF metadata (hyperref)
%\newcommand{\mytitle}{TITEL}  %% also used for PDF metadata (hyperref)
%\newcommand{\mysubtitle}{UNTERTITEL}  %% also used for PDF metadata (hyperref)
%\newcommand{\mysubject}{GEBIET/FELD}  %% also used for PDF metadata (hyperref)
%\newcommand{\mykeywords}{SCHLÜSSELWÖRTER}  %% also used for PDF metadata (hyperref)
%
%%% this information is used only for generating the title page:
%\newcommand{\myworktitle}{Masterarbeit}  %% official type of work like ``Master theses''
%\newcommand{\mygrade}{Master of Science} %% title you are getting with this work like ``Master of ...''
%\newcommand{\mystudy}{Psychologie} %% your study like ``Arts''
%\newcommand{\myuniversity}{Otto-von-Guericke-Universit\"at} %% your university/school
%\newcommand{\myfaculty}{Fakult\"at f\"ur Naturwissenschaften} %% affiliation
%\newcommand{\myinstitute}{Institut f\"ur Psychologie II} %% affiliation
%\newcommand{\mysupervisor}{Dr.~Some Body} %% your supervisor
%\newcommand{\mycosupervisor}{Dr.~Some Body Else} %% your supervisor
%\newcommand{\mysubmissionmonth}{Oktober} %% month you are handing in
%\newcommand{\mysubmissionyear}{2015} %% year you are handing in
%\newcommand{\mysubmissiontown}{Magdeburg} %% town of handing in
%
%%% additional information for generic_documentation title page
%\newcommand{\myid}{1234567} %% Matrikelnummer




%% BibLaTeX-settings: (see biblatex reference for further description)
\newcommand{\mybiblatexstyle}{authoryear}
%% e.g., "alphabetic", "authoryear", ...
%% The biblatex style which is being used for referencing. See
%% biblatex documentation for further details and more values.
%%
%% CAUTION: if you change the style, please check for (in)compatible
%%          "biblatex" package options in the file
%%          "template/preamble.tex"! For example: "alphabetic" does
%%          not have an option "dashed=..." and causes an error if it
%%          does not get removed from the list of options.

\newcommand{\mybiblatexdashed}{false}  %% "true" or "false"
%% If true: replace recurring reference authors with a dash.

\newcommand{\mybiblatexbackref}{true}  %% "true" or "false"
%% If true: create backward links from reference to citations.

\newcommand{\mybiblatexfile}{bib/references-biblatex.bib}
%% Name of the biblatex file that holds the references.

\newcommand{\mydispositioncolor}{0,0,0}
%% e.g., "30,103,182" (blue/turquois), "0,0,0" (black), ...
%% Color of the headings and so forth in RGB (red,green,blue) values.
%% NOTE: if you are using "0,0,0" for black, printers might still
%%       recognize pages as color pages. In case this is a problem
%%       (paying for color print-outs vs. paying for b/w-printouts)
%%       please edit file "template/preamble.tex" and change
%%       "\definecolor{DispositionColor}{RGB}{\mydispositioncolor}"
%%       to "\definecolor{DispositionColor}{gray}{0}" and thus
%%       overwriting the value of \mydispositioncolor above.

\newcommand{\mycolorlinks}{true}  %% "true" or "false"
%% Enables or disables colored links (hyperref package).

\newcommand{\mytitlepage}{template/title_thesis_ovgu_psych}
%% Your own or one of following pre-defined title pages:
%% "template/title_plain_maketitle": simple maketitle page
%% "template/title_Diplomarbeit_KF_Uni_Graz.tex": fancy (german) title page for KF Uni Graz
%% "template/title_Thesis_TU_Graz":
%%             titlepage for Graz University of Technology (correct
%%             (old?) Corporate Design) by Karl Voit (2012)
%% "template/title_Thesis_TU_Graz_-_kazemakase":
%%             titlepage for Graz University of Technology
%%             (correct new Corporate Design) by kazemakase (2013):
%%             see https://github.com/novoid/LaTeX-KOMA-template/issues/5
%% "template/title_VWA": titlepage for Vorwissenschaftliche Arbeit

\newcommand{\mytodonotesoptions}{}
%% e.g., "" (empty), "disable", ...
%% Options for the todonotes-package. If "disable", all todonotes will
%% be hidden (including listoftodos).

%% Load main settings for document preamble:
\input{template/preamble}%% DO NOT REMOVE THIS LINE!

\setboolean{myaddcolophon}{false}  %% "true" or "false"
%% If set to "true": a colophon (with notes about this document
%% template, LaTeX, ...) is added after the title page.
%% Please do not set to "false" without a good reason. The colophon
%% helps your readers to get in touch with LaTeX and to find this template.

\setboolean{myaddlistoftodos}{true}  %% "true" or "false"
%% If set to "true": the current list of open todos is added after the
%% table of contents. If \mytodonotesoptions is set to "disable", no
%% list of todos is added, independent of this setting here.




%% ========================================================================
%%%% MISC command definitions
%% ========================================================================
\input{template/mycommands}

%% ========================================================================
%%%% Typographic settings
%% ========================================================================
\input{template/typographic_settings}


%% ========================================================================
%%%% MISC usepackages
%% ========================================================================

%% ... it's OK to put here your own usepackage commands ...




%% ========================================================================
%%%% MISC self-defined commands and settings
%% ========================================================================

%% ... it's OK to put here your own newcommand/newenvironment-definitions ...




\newcommand{\myLaT}{\LaTeX{}@TUG\xspace} %% LaTeX@TUG text "logo"

\hyphenation{ex-am-ple hy-phen-ate}  %% in order to use German umlauts
%% here (Ver-\"of-fent-li-chung), you have to check for
%% activated \usepackage[T1]{fontenc} in the preamble

%% override default language of babel: (be sure to know, what you're
%% doing here)
%\selectlanguage{american}
%\selectlanguage{ngerman}

%% ========================================================================
%%%% Templates
%% ========================================================================

%% template for inserting figures:
% \myfig{}%% filename
%       {}%% width/height
%       {}%% caption
%       {}%% optional (short) caption for list of figures
%       {fig:}%% label

%% acronyms in small caps: \myacro{UNESCO}


\input{template/pdf_settings}  %% should be *last* definitions in preamble!
%% ========================================================================
%%%% begin{document}
%% ========================================================================
\begin{document}

\frontmatter                    %% KOMA: roman page numbers and such; only available in scrbook

\input{template/colophon}                %% defines information about editor, LaTeX, font, ...

%% Choose your desired title page:
\input{\mytitlepage}            %% include title page

%% if myaddlistoftodos is set to "true", the current list of open todos is added:
\ifthenelse{\boolean{myaddlistoftodos}}{
  \newpage\listoftodos\newpage  %% handy if you are using todonotes with \todo{}
}{}                             %% with todonotes-package option "disable" you can get rid of any todo in the output

%%%% Time-stamp: <2013-01-02 14:43:58 vk>
%% ========================================================================
%%%% Disclaimer
%% ========================================================================
%%
%% created by
%%
%%      Karl Voit
%%

\section*{Statutory Declaration}

I declare that I have authored this thesis independently, that I have
not used other than the declared sources/resources, and that I have
explicitly marked all material which has been quoted either literally
or by content from the used sources.

\vfill

%% definition of the block tat contains date and signature
\newcommand{\mysignatureblock}[3]{%
  %% Sorry, this is a "bit" of a hack. Maybe someone knows a more elegant method?
  \begin{tabular}{llp{2em}l} 
  #1 & \hspace{3cm}        & & \hspace{56mm} \\\cline{2-2}\cline{4-4}
     &                     & & \\[-3mm]
     & {\footnotesize #2}  & & {\footnotesize #3}
  \end{tabular}
}

\mysignatureblock{Magdeburg,}{Date}{Signature}

\vfill
\vfill
\vfill
\vfill

\section*{Ehrenwörtliche Versicherung}

\foreignlanguage{ngerman}{%
Hiermit versichere ich, dass ich die vorliegende Arbeit selbständig und nur unter Benutzung der
angegebenen Literatur- und Hilfsmittel angefertigt habe. Wörtlich übernommene Sätze und
Satzteile aus anderen Druckwerken oder aus Internetpublikationen sind als Zitat belegt, andere
Anlehnungen hinsichtlich Aussage und Umfang unter Angabe der Quelle kenntlich gemacht. Die
Arbeit wurde in gleicher oder ähnlicher Form in keiner anderen Lehrveranstaltung als
Leistungsnachweis eingereicht.}

\foreignlanguage{ngerman}{%
Ich bin darüber unterrichtet, dass die Lehrenden angewiesen sind, schriftliche Arbeiten zu
überprüfen, und dass ein Vergehen eine Meldung beim Prüfungsausschuss der Fakultät zur Folge
hat, die im schlimmsten Fall zum Ausschluss aus der Universität führen kann.
}

\vfill

\mysignatureblock{Magdeburg, am}{Datum}{Unterschrift}


\newpage
\newpage

%%% Local Variables: 
%%% mode: latex
%%% mode: auto-fill
%%% mode: flyspell
%%% TeX-master: "../main"
%%% End: 
  %% Statutory Declaration

%% include the abstract without chapter number but include it on table of contents:
\cleardoublepage
%\addcontentsline{toc}{chapter}{Zusammenfassung}
%\chapter*{Zusammmenfassung}
\chapter*{Abstract}
\addcontentsline{toc}{chapter}{Abstract}
\label{cha:abstract}


This is a placeholder for the abstract. It summarizes the whole thesis
to give a very short overview. Usually, this the abstract is written
when the whole thesis text is finished.



%\glsresetall %% all glossary entries should be used in long form (again)
%% vim:foldmethod=expr
%% vim:fde=getline(v\:lnum)=~'^%%%%\ .\\+'?'>1'\:'='
%%% Local Variables:
%%% mode: latex
%%% mode: auto-fill
%%% mode: flyspell
%%% eval: (ispell-change-dictionary "en_US")
%%% TeX-master: "main"
%%% End:
              %% Abstract

\tableofcontents                %% this produces the table of contents - you might have guessed :-)

% enable the following if you really want a list of figure -- usually not needed
%\listoffigures

\mainmatter                     %% KOMA: marks main part using arabic page numbers and such; only available in scrbook

% some example chapters -- remove when you have your own written
\input{examples/example-short-chapter}   %% remove this line to get rid of the example chapter
\input{examples/example-style-chapter}   %% remove this line to get rid of the style chapter

%% include tex file chapters here:
% \input{thanks}                %% this is a suggestion: you have to create this file on demand
% \input{foreword}              %% this is a suggestion: you have to create this file on demand
% \include{introduction}        %% this is a suggestion: you have to create this file on demand
% \include{problem}             %% this is a suggestion: you have to create this file on demand
% \include{solution}            %% this is a suggestion: you have to create this file on demand
% \include{evaluation}          %% this is a suggestion: you have to create this file on demand
% \include{outlook}             %% this is a suggestion: you have to create this file on demand

% in some cases an appendix is desired/needs -- uncomment if needed
%\appendix                       %% closes main document, appendix follows until end; only available in book-classes
%\addpart*{Appendix}             %% adding Appendix to tableofcontents

% this will generate the references -- pretty much always needed
\printbibliography              %% remove, if using BibTeX instead of biblatex
% \include{further_ressources}  %% this is a suggestion: you have to create this file on demand


%%%% end{document}
\end{document}
%% vim:foldmethod=expr
%% vim:fde=getline(v\:lnum)=~'^%%%%\ .\\+'?'>1'\:'='
%%% Local Variables:
%%% mode: latex
%%% mode: auto-fill
%%% mode: flyspell
%%% eval: (ispell-change-dictionary "en_US")
%%% TeX-master: "main"
%%% End:
