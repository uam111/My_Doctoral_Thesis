% First define the language:
%% "english,ngerman", "ngerman,english", ...
%% NOTE: The *last* language is the active one!
%% See babel documentation for further details.
%\newcommand{\mylanguage}{american,ngerman}
\newcommand{\mylanguage}{ngerman, american}

%% general metadata:
\newcommand{\myauthor}{Ayan Sengupta}  %% also used for PDF metadata (hyperref)
\newcommand{\myid}{206221} %% Matrikelnummer/Student ID
\newcommand{\mytitle}{TITLE}  %% also used for PDF metadata (hyperref)
\newcommand{\mysubtitle}{SUBTITLE}  %% also used for PDF metadata (hyperref)
\newcommand{\mysubject}{SUBJECT}  %% also used for PDF metadata (hyperref)
\newcommand{\mykeywords}{KEYWORDS}  %% also used for PDF metadata (hyperref)
\newcommand{\mysupervisor}{Dr.Stefan Pollmann} %% your supervisor
\newcommand{\mycosupervisor}{Dr.Michael Hanke} %% your supervisor
\newcommand{\mysubmissionmonth}{November} %% month you are handing in
\newcommand{\mysubmissionyear}{2015} %% year you are handing in
\newcommand{\mysubmissiontown}{Magdeburg} %% town of handing in

% 1st in English: German below (un)comment according to needs
%% this information is used only for generating the title page:
%\newcommand{\myworktitle}{Master's Thesis}  %% official type of work like ``Master theses''
%\newcommand{\mygrade}{Master of Science} %% title you are getting with this work like ``Master of ...''
%\newcommand{\mystudy}{Psychology} %% your study like ``Arts''
%\newcommand{\myuniversity}{Otto-von-Guericke-University} %% your university/school
%\newcommand{\myfaculty}{Fakulty of Natural Sciences} %% affiliation
%\newcommand{\myinstitute}{Institute of Psychology II} %% affiliation

%% 2nd German
%%% this information is used only for generating the title page:
\newcommand{\myworktitle}{Doctoral Thesis}  %% official type of work like ``Master theses''
\newcommand{\mygrade}{Doctor rerum naturalium} %% title you are getting with this work like ``Master of ...''
\newcommand{\mystudy}{Experimental Psychology} %% your study like ``Arts''
\newcommand{\myuniversity}{Otto-von-Guericke-Universit\"at} %% your university/school
\newcommand{\myfaculty}{Fakult\"at f\"ur Naturwissenschaften} %% affiliation
\newcommand{\myinstitute}{Institut f\"ur Psychologie II} %% affiliation
\newcommand{\purl}[2]{#1\footnote{\url{#2}}}

\newcommand{\sevenT}{\unit[7]{Tesla}}
\newcommand{\threeT}{\unit[3]{Tesla}}
\newcommand{\mm}[1]{\unit[#1]{mm}}
\newcommand{\seconds}[1]{\unit[#1]{s}}
\newcommand{\bandofinterest}{\unit[$\approx$5-8]{mm}}
%% Miscellaneous Latin abbreviations emphasizing
%% according to
%% http://en.wikipedia.org/wiki/List_of_Latin_phrases_(C-E)#endnote_egie
%% ``American style guides tend to recommend that "e.g."  and "i.e."
%% should generally be followed by a comma, just as "for example" and
%% "that is" would be; UK style tends to omit the comma''
\newcommand{\ie}[0]{\emph{i.e.},\ }
\newcommand{\eg}[0]{\emph{e.g.},\ }
\newcommand{\etc}[0]{\emph{etc.}}

%% Unified references
\newcommand{\fig}[1]{{Figure~\ref{fig:#1}}}

% Cross-referencing (comment out if forbidden)
\newcommand{\msec}[1]{Section~\ref{sec:#1}}
\newcommand{\mtab}[1]{Table~\ref{tab:#1}}
\newcommand{\msecs}[2]{Sections~\ref{sec:#1} and~\ref{sec:#2}}
