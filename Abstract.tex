\specialhead{Abstract}

A decade after it was shown that the orientation of visual grating stimuli
can be decoded from human visual cortex activity by means of multivariate
pattern classification of BOLD fMRI data, numerous studies have investigated
which aspects of neuronal activity are reflected in BOLD response patterns
and are accessible for decoding. However, even for orientation decoding, it
remains inconclusive what minimum spatial acquisition resolution is required
--- or optimal --- for decoding analyses.
%
The present study is the first to provide empirical ultra high-field fMRI
data recorded at four spatial resolutions (\mm{0.8}, \mm{1.4}, \mm{2}, and
\mm{3} isotropic voxel size) on this topic --- in order to test hypotheses on
the strength and spatial scale of orientation discriminating signals.
%
We present evidence, in line with predictions from previous simulation
studies, that a spatial resolution of \mm{$\approx$2.5} isotropic voxel size
is optimal for orientation decoding. Moreover, higher-resolution scans with
subsequent down-sampling or low-pass filtering yield no benefit over scans
natively recorded in the corresponding lower resolution. The
orientation-related signal in the BOLD fMRI data is spatially broadband in nature,
includes both high spatial frequency components, as well as large-scale
biases previously proposed in the literature. We find no evidence, however,
for ``hyperacuity'' or an aliased low-spatial-frequency representation of
high-frequency signals with columnar origin.

